\documentclass{article}
\usepackage[utf8]{inputenc}
\usepackage{datetime}
\usepackage{enumerate}
\usepackage{textcomp}
\usepackage{amsmath}
\usepackage{amssymb}
\usepackage{tikz}
\usetikzlibrary{arrows,shapes,backgrounds}

\usepackage{titlesec}
\newcommand{\sectionbreak}{\clearpage}

\title{\bf \Large ASSIGNMENT 2}
\author{Xinhao Luo} 
\date{\today}

\def\math#1{$#1$}

\setlength{\textheight}{8.5in}
\setlength{\textwidth}{6.5in}
\setlength{\oddsidemargin}{0in}
\setlength{\evensidemargin}{0in}
\voffset0.0in

\begin{document}
\maketitle
\medskip

\section{Problem 1.6}

\begin{itemize}
    \item P1 
    \begin{itemize}
        \item CPU Clock Cycle
            \begin{equation}
                \begin{split}
                   (1 \times 10\% \times 10^6) + (2 \times 20\% \times 10^6) + (3 \times 50\% \times 10^6) + (3 \times 20\% \times 10^6) = 2.6 \times 10^6
                \end{split}
            \end{equation}
        \item CPU Time
            \begin{equation}
                \begin{split}
                  \frac{2.6 \times 10^6}{2.5 GHz} = 1.04 \times 10^{-3} s
                \end{split}
            \end{equation}
    \end{itemize}
    \item P2
        \begin{itemize}
        \item CPU Clock Cycle
            \begin{equation}
                \begin{split}
                   (2 \times 10\% \times 10^6) + (2 \times 20\% \times 10^6) + (2 \times 50\% \times 10^6) + (2 \times 20\% \times 10^6) = 2 \times 10^6
                \end{split}
            \end{equation}
        \item CPU Time
            \begin{equation}
                \begin{split}
                  \frac{2 \times 10^6}{3 GHz} = 6.67 \times 10^{-4} s
                \end{split}
            \end{equation}
    \end{itemize}
\end{itemize}

P2 implementation is faster

\begin{enumerate}[(a)]
    \item Global CPI
        \begin{itemize}
            \item P1 
                \begin{equation}
                    \begin{split}
                       \frac{2.6 \times 10^6}{10^6} = 2.6
                    \end{split}
                \end{equation}
            \item P2
                \begin{equation}
                    \begin{split}
                       \frac{2 \times 10^6}{10^6} = 2
                    \end{split}
                \end{equation}
        \end{itemize}
    \item Clock Cycles
        \begin{itemize}
            \item P1: \math{2.6 \times 10^6}
            \item P2: \math{2 \times 10^6}
        \end{itemize}
\end{enumerate}

\section{Problem 1.9}
\begin{enumerate}[(1)]
    \item Execution Time
        \begin{itemize}
            \item [1 Processor]
                 \begin{equation}
                    \begin{split}
                       \frac{(1 \times 2.56 \times 10^9) + (12 \times 1.28 \times 10^9) + (5 \times 2.56 \times 10^8)}{2 GHz} = 9.6s
                    \end{split}
                \end{equation}
            \item [2 Processors]
                \begin{equation}
                    \begin{split}
                       \frac{\frac{(1 \times 2.56 \times 10^9) + (12 \times 1.28 \times 10^9)}{0.7 \times 2} + (5 \times 2.56 \times 10^8)}{2 GHz} = 7.04s
                    \end{split}
                \end{equation}
            Speed Up \math{9.6 / 7.04 = 1.36 }
            \item [4 Processors]
                \begin{equation}
                    \begin{split}
                       \frac{\frac{(1 \times 2.56 \times 10^9) + (12 \times 1.28 \times 10^9)}{0.7 \times 4} + (5 \times 2.56 \times 10^8)}{2 GHz} = 3.84s
                    \end{split}
                \end{equation}
              Speed Up \math{9.6 / 3.84 = 2.5}
            \item [8 Processors]
                \begin{equation}
                    \begin{split}
                       \frac{\frac{(1 \times 2.56 \times 10^9) + (12 \times 1.28 \times 10^9)}{0.7 \times 8} + (5 \times 2.56 \times 10^8)}{2 GHz} = 2.24s
                    \end{split}
                \end{equation}
                Speed Up \math{9.6 / 2.24 = 4.29}
        \end{itemize}
    \item Arithmetic Doubled
        \begin{itemize}
            \item [1 Processor]
                 \begin{equation}
                    \begin{split}
                       9.6s + \frac{(1 \times 2.56 \times 10^9)}{2 GHz} = 10.88s
                    \end{split}
                \end{equation}
                Slowed down \math{10.88 / 9.6 = 1.13}
            \item [2 Processors]
                \begin{equation}
                    \begin{split}
                       7.04s + \frac{(1 \times 2.56 \times 10^9)}{0.7 \times 2 \times 2 GHz} = 7.95s
                    \end{split}
                \end{equation}
                Slowed down \math{7.95 / 7.04 = 1.13}
            \item [4 Processors]
                \begin{equation}
                    \begin{split}
                        3.84s + \frac{(1 \times 2.56 \times 10^9)}{0.7 \times 4 \times 2 GHz} = 4.30s
                    \end{split}
                \end{equation}
              Slowed down \math{4.30 / 3.84 = 1.12}
            \item [8 Processors]
                \begin{equation}
                    \begin{split}
                        2.24s + \frac{(1 \times 2.56 \times 10^9)}{0.7 \times 8 \times 2 GHz} = 2.47s
                    \end{split}
                \end{equation}
                Slowed down \math{2.47 / 2.24 = 1.10}
        \end{itemize}
    \item Reduced Load function
        \begin{equation}
            \begin{split}
               \frac{(1 \times 2.56 \times 10^9) + (1.28 \times x \times 10^9)}{0.7 \times 4} &= (1 \times 2.56 \times 10^9) + (1.28 \times x \times 10^9) \\
               x &= 3 
            \end{split}
        \end{equation}
        When the number of load instruction reduced to \math{\frac{|3-12|}{12} = 75\%}, a single processor will have same performance as 4 processors.
\end{enumerate}

\section{Problem 1.12}
\begin{enumerate}[(1)]
    \item Fallacy
        \begin{itemize}
            \item [P1 CPU Time]
             \begin{equation}
                    \begin{split}
                       \frac{0.9 \times 5 \times 10^9}{4GHz} = 1.125
                    \end{split}
                \end{equation}
            \item [P2 CPU Time]
                \begin{equation}
                    \begin{split}
                       \frac{0.75 \times 1 \times 10^9}{3GHz} = 0.25
                    \end{split}
                \end{equation}
        \end{itemize}
        It is not true for both P1 and P2
    \item  \begin{equation}
                \begin{split}
                   \frac{0.9 \times 1 \times 10^9}{4GHz} &= \frac{0.75 \times x}{3GHz} \\
                   x &= 0.9 \times 10^9
                \end{split}
            \end{equation}
            P2 can execute \math{0.9 \times 10^9} instruction at the same time.
\end{enumerate}

\section{Problem 1.14}

\begin{enumerate}[(1)]
    \item \math{\text{Execution Time} = \frac{\text{Clock Cycles}}{\text{Clock rate}}}
        \begin{equation}
            \begin{split}
               \frac{(1 \times 5 \times 10^7) + (1 \times 1.1 \times 10^8) + (4 \times 8 \times 10^7) + (2 \times 1.6 \times 10^7)}{2GHz} = 2.56 \times 10^{-1} s\\
               1.28 \times 10^{-1}s = \frac{(x \times 5 \times 10^7) + (1 \times 1.1 \times 10^8) + (4 \times 8 \times 10^7) + (2 \times 1.6 \times 10^7)}{2GHz} \\
               x = -4.12
            \end{split}
        \end{equation}
        However, a negative CPI is not possible. 
    \item 
        \begin{equation}
            \begin{split}
               1.28 \times 10^{-1}s = \frac{(1 \times 5 \times 10^7) + (1 \times 1.1 \times 10^8) + (x \times 8 \times 10^7) + (2 \times 1.6 \times 10^7)}{2GHz} \\
               x = 0.8
            \end{split}
        \end{equation}
    \item  \begin{equation}
                \begin{split}
                   \frac{(1 \times (1 - 0.4) \times 5 \times 10^7) + (1 \times (1 - 0.4) \times 1.1 \times 10^8) + (4 \times (1 - 0.3) \times 8 \times 10^7) + (2 \times (1 - 0.3) \times 1.6 \times 10^7)}{2GHz} \\
                   = 1.712 \times 10^{-1} s
                \end{split}
            \end{equation}
        Speed up \math{\frac{2.56 \times 10^{-1}}{1.712 \times 10^{-1}} = 1.49}
\end{enumerate}

\end{document}
